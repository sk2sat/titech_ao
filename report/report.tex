\documentclass[10pt,a4j]{jsarticle}

\title{\vspace{-4cm}OSが動作するx86エミュレータの開発}
\author{坂本優太}
\date{}

\begin{document}
\maketitle

\renewcommand{\abstractname}{背景}
\begin{abstract}
% 僕は,中学生の時に\cite{30days-osdev}をきっかけにコンピュータ・サイエンスの低レイヤの分野に興味を持った.
% また,僕は高校1年生の時にセキュリティ・キャンプ全国大会2016
% \footnote{本における将来の高度IT人材となり得る優れた人材の発掘と育成を目的とした独立行政法人情報処理推進機構(IPA)の事業のメインイベントのうち,2016年に行われたもの.}
% に参加し,低レイヤにより強い興味を持つようになった.
% その後,\cite{learn-x86-by-emu}を読んだが,
% \cite{30days-osdev}で作る簡易的なOSである「はりぼてOS」を動かしたいと思い
% ↑ ここらへんは志望理由の方が良い?
\end{abstract}

\section{成果の具体的な内容}
\subsection{エミュレータの仕組み}
\subsection{実装}
・16bitと32bit分ける
・ディスプレイのエミュレーション

\subsection{サイボウズ・ラボユースでの成果}
・vmからemuに
・SDMたべる
・ログの出力
・セグメンテーションの実装
・実行テスト環境
\subsection{v2の実装}
・新たな実装
・実装の見通しの悪さを改善
・ラムダ式

\section{単独の成果か否か}
単独の成果である.
実装に当たっては\cite{learn-x86-by-emu}のプログラムを参考にはしたものの,設計を変更し,機能も大幅に増えたオリジナルのプログラムとなっている.
設計・実装はすべて1人で行った.
ただし,サイボウズ・ラボユース採択期間中は,ラボユース内のC++勉強会に参加した他,メンターの方に絶版となっていた\cite{read-486}を貸して頂いたり,数回コードレビューを頂いたりした.
コードレビューで指摘されたのはC++の書き方に関するものであり,これによる設計やロジックの変更は生じなかった.
\footnote{コードレビューによる変更は"[FIX] from code review"というコミットで行われている.}


\begin{thebibliography}{99}
	\bibitem{30days-osdev} 30日でできる! OS自作入門
	\bibitem{learn-x86-by-emu} 自作エミュレータで学ぶx86アーキテクチャ - コンピュータが動く仕組みを徹底理解!
	\bibitem{SDM} Intel® 64 and IA-32 Architectures Software Developer’s Manual vol 1,2,3,4
	\bibitem{read-486} はじめて読む486 - 32ビットコンピュータをやさしく語る
\end{thebibliography}

\end{document}
