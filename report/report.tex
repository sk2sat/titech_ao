\documentclass[10pt,a4j]{jsarticle}

\usepackage[dvipdfmx]{hyperref}

\title{\vspace{-4cm}OSが動作するx86エミュレータの開発}
\author{坂本優太}
\date{}

\begin{document}
\maketitle

%\renewcommand{\abstractname}{背景}
%\begin{abstract}
% 僕は,中学生の時に\cite{30days-osdev}をきっかけにコンピュータ・サイエンスの低レイヤの分野に興味を持った.
% また,僕は高校1年生の時にセキュリティ・キャンプ全国大会2016
% \footnote{本における将来の高度IT人材となり得る優れた人材の発掘と育成を目的とした独立行政法人情報処理推進機構(IPA)の事業のメインイベントのうち,2016年に行われたもの.}
% に参加し,低レイヤにより強い興味を持つようになった.
% その後,\cite{learn-x86-by-emu}を読んだが,
% \cite{30days-osdev}で作る簡易的なOSである「はりぼてOS」を動かしたいと思い
% ↑ ここらへんは志望理由の方が良い?
%\end{abstract}

\section{はじめに}
エミュレータとは,コンピュータの動作をエミュレート,つまり模倣するプログラムのことで,代表的なものとしては
QEMU\footnote{Fabrice Bellardが中心となって開発しているオープンソースのエミュレータ.動的バイナリ変換(Dynamic Binary Translation)という機能により高速なエミュレートを行うことができる.}
やBochs\footnote{オープンソースのPC/AT互換機のエミュレータ.}などのものがある.


\section{成果の具体的な内容}

\subsection{実装}
%・16bitと32bit分ける
%・ディスプレイのエミュレーション
\subsubsection{コンピュータの仕組み}
エミュレータはコンピュータをエミュレートするプログラムである.
そのため,エミュレータを作るためにはまずコンピュータの仕組みを知る必要がある.

現代的なコンピュータを構成する要素には,
\begin{itemize}
	\item CPU
	\item メモリ
	\item 外部装置
\end{itemize}
といったものがある.

これらの要素のうち,最も重要なのはCPUだ.
CPUは命令を実行する装置で,
CPUが実行する命令は,メモリの内部にデータとして保存されている.
この命令はすべて0と1の2進数で表されており

コンピュータが起動すると,まず初めにROMに保存されているBIOS
\footnote{Basic Input/Output System}
というプログラムが実行が始まる.
BIOSは,HDDなどの外部記憶からブートローダという小さなプログラムをメモリに読み込み,実行する.
そして,ブートローダは外部記憶からOSを読み込み,実行する.

\subsubsection{エミュレータの仕組み}
\cite{learn-x86-by-emu}はエミュレータを自作することでコンピュータが動く仕組みを理解するという趣旨の本だが,
これはつまりエミュレータを作るためにはコンピュータの仕組みを学ぶ必要があるということでもある.



\subsubsection{動作モードの実装}

\subsubsection{ディスプレイの実装}


\subsection{サイボウズ・ラボユースでの成果}
%・vmからemuに
%・SDMたべる
%・ログの出力
%・セグメンテーションの実装
%・実行テスト環境

\subsection{v2の実装}
%・新たな実装
%・実装の見通しの悪さを改善
%・ラムダ式

\section{単独の成果か否か}
単独の成果である.
実装に当たっては\cite{learn-x86-by-emu}のプログラムを参考にはしたものの,設計を変更し,機能も大幅に増えたオリジナルのプログラムとなっている.
設計・実装はすべて1人で行った.
ただし,サイボウズ・ラボユース採択期間中は,ラボユース内のC++勉強会に参加した他,メンターの方に絶版となっていた\cite{read-486}を貸して頂いたり,数回コードレビューを頂いたりした.
コードレビューで指摘されたのはC++の書き方に関するものであり,これによる設計やロジックの変更は生じなかった.
\footnote{コードレビューによる変更は"[FIX] from code review"というコミットで行われている.}


\begin{thebibliography}{99}
	\bibitem{30days-osdev} 30日でできる! OS自作入門
	\bibitem{learn-x86-by-emu} 自作エミュレータで学ぶx86アーキテクチャ - コンピュータが動く仕組みを徹底理解!
	\bibitem{SDM} Intel® 64 and IA-32 Architectures Software Developer’s Manual vol 1,2,3,4
	\bibitem{read-486} はじめて読む486 - 32ビットコンピュータをやさしく語る
\end{thebibliography}

\end{document}
